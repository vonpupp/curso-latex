\documentclass[a4paper]{article}
\usepackage{verbatim}
\usepackage{listings}

% It's a good practice to include all newcommands at the beginning of the document
\newcommand{\RFB}{Republica Federativa do Brasil}
\newcommand{\eng}[1]{\textit{#1}}

%\newcommand{\commandname}[# or arguments]{•}
\newcommand{\lista}[2]{
	\begin{itemize}
		\item #1
		\item #2
	\end{itemize}
	}
	
\newcommand{\arrow}[2]{
	\begin{itemize}
		#1 \rightarrow #2
	\end{itemize}
	}
	
\newcommand{\twoarrow}[3]{
	\begin{itemize}
		#1 \rightarrow #2 \rightarrow #3
	\end{itemize}
	}
	
\begin{document}
	\RFB
	
	\eng{School}
	
	\lista{a} {b}
	
	\arrow{c} {d}
	
	\twoarrow{e} {f} {g}
	\\
	
	text\\
	\texttt{               text true type}\\
	\verb|      a\\
	\\
	\begin{verbatim}
    def initparser(self, ModelClass):#, ParserClass):
        # Public
        # Init structures
        if ModelClass is rbm1.ReqBoxModel:
            ParserClass = rfp1.ReqBoxFileParser
        elif ModelClass is rbm2.ReqBoxModelNG:
            ParserClass = rfp2.ReqBoxFileParserNG
        self.model = ModelClass(ParserClass)
        #self.model = rbm.ReqBoxModel(ParserClass)
        if ParserClass is rfp2.ReqBoxFileParserNG:
            self.model.fp.importsdir = './data/'
	\end{verbatim}
	
	
	\begin{lstlisting}[language=python]
    def initparser(self, ModelClass):#, ParserClass):
        # Public
        # Init structures
        if ModelClass is rbm1.ReqBoxModel:
            ParserClass = rfp1.ReqBoxFileParser
        elif ModelClass is rbm2.ReqBoxModelNG:
            ParserClass = rfp2.ReqBoxFileParserNG
        self.model = ModelClass(ParserClass)
        #self.model = rbm.ReqBoxModel(ParserClass)
        if ParserClass is rfp2.ReqBoxFileParserNG:
            self.model.fp.importsdir = './data/'
	\end{lstlisting}
	
	To manually compile use:
	\begin{lstlisting}[language=bash]
	latex aula03.tex
	pdflatex aula03
	\end{lstlisting}
\end{document}