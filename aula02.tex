\documentclass[a4paper]{article}
\usepackage[utf8]{inputenc}
\usepackage[brazil]{babel} % Esse pacote e para hifenizacao
\usepackage[T1]{fontenc} % Permite copiar
\usepackage{verbatim}
\usepackage{amsmath}
\usepackage{graphicx}
\usepackage{float}

\author{Albert}
\title{Aula 1}

\begin{document}
\maketitle
O \emph{Muro de Berlim (em alemão Berliner Mauer)} era uma barreira física, construída pela República Democrática Alemã (Alemanha Oriental) durante a Guerra Fria, que circundava toda a Berlim Ocidental, separando-a da Alemanha Oriental, incluindo Berlim Oriental. Este muro, além de dividir a cidade de Berlim ao meio, simbolizava a divisão do mundo em dois blocos ou partes: República Federal da Alemanha (RFA), que era constituído pelos países capitalistas encabeçados pelos Estados Unidos; e República Democrática Alemã (RDA), constituído pelos países socialistas simpatizantes do regime soviético. Construído na madrugada de 13 de Agosto de 1961, dele faziam parte 66,5 km de gradeamento metálico, 302 torres de observação, 127 redes metálicas electrificadas com alarme e 255 pistas de corrida para ferozes cães de guarda. Este muro era patrulhado por militares da Alemanha Oriental com ordens de atirar para matar (a célebre Schießbefehl ou "Ordem 101") os que tentassem escapar, o que provocou a morte a 80 pessoas identificadas, 112 ficaram feridas e milhares aprisionadas nas diversas tentativas.\\ \\
A distinta e muito mais longa fronteira interna alemã demarcava a fronteira entre a Alemanha Oriental e a Alemanha Ocidental. Ambas as fronteiras passaram a simbolizar a chamada "cortina de ferro" entre a Europa Ocidental e o Bloco de Leste.

Antes da construção do Muro, 3,5 milhões de alemães orientais tinham evitado as restrições de emigração do Leste e fugiram para a Alemanha Ocidental, muitos ao longo da fronteira entre Berlim Oriental e Ocidental. Durante sua existência, entre 1961 e 1989, o Muro quase parou todos os movimentos de emigração e separou a Alemanha Oriental de Berlim Ocidental por mais de um quarto de século.\footnote{http://www.time.com/time/magazine/article/0,9171,959058,00.html}

\section*{Antecedentes}
Após a Segunda Guerra Mundial na Europa, o que restou da Alemanha nazista a oeste da linha Oder-Neisse foi dividido em quatro zonas de ocupação (por Acordo de Potsdam), cada um controlado por uma das quatro potências aliadas: os Estados Unidos, o Reino Unido, a França e a União Soviética. A capital, Berlim, enquanto a sede do Conselho de Controle Aliado, foi igualmente dividida em quatro sectores, apesar da cidade estar situada bem no interior da zona soviética.[2] Em dois anos, ocorreram divisões entre os soviéticos e as outras potências de ocupação, incluindo a recusa dos soviéticos aos planos de reconstrução para uma Alemanha pós-guerra auto-suficiente e de uma contabilidade detalhada das instalações industriais e infra-estrutura já removidas pelos soviéticos.[3] Reino Unido, França, Estados Unidos e os países do Benelux se reuniram para mais tarde transformar as zonas não-soviéticas do país em zonas de reconstrução e aprovar a ampliação do Plano Marshall para a reconstrução da Europa para a Alemanha.
\section*{Construção do muro}
\verb \section*{*} e usado para definir uma seccao
\section*{Reações}

\section*{Este artigo faz parte de uma série}
\begin{itemize}
\item República de Weimar (1919–33) - http://pt.wikipedia.org/wiki/Rep%C3%BAblica_de_Weimar
\item Berlim em 1920
\item Grande decreto de Berlim
\end{itemize}
\begin{enumerate}
\item Alemanha Nazi (1933–45)
\item Welthauptstadt Germania
\end{enumerate}

\section*{Resources}
Editado com texmaker
www.ctan.org -> lshort

\section*{Math}
Inline math is edited as this (Tex way): $a^2 = b^2 + c^2$ or as a Latex way like this:
\begin{equation}
	a^2 + b^2 = c^2 \label{eq:pitagoras1}
\end{equation}

To use references. Lets asssume the equation:
\begin{displaymath}
	a^2 + b^2 = c^2 \label{eq:pitagoras2}
\end{displaymath}
According to the reference \ref{eq:pitagoras2}


\[
 z = \overbrace{
   \underbrace{x}_\text{real} +
   \underbrace{iy}_\text{imaginary}
  }^\text{complex number}
\]

\begin{figure}
\includegraphics[width=0.5\textwidth]{funny-pic}\caption{Funny pic}
\end{figure}

\begin{equation}
\frac{a}{b} = \frac{c}{d}
\end{equation}
\begin{displaymath}
	a^2 = b^2 + c^2 (not numbered)
\end{displaymath}
%\end{displaymath}
\end{document}